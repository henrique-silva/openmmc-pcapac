\documentclass[a4paper,
               %boxit,
               %titlepage,   % separate title page
               %refpage      % separate references
              ]{jacow}
%
% CHANGE SEQUENCE OF GRAPHICS EXTENSION TO BE EMBEDDED
% ----------------------------------------------------
% test for XeTeX where the sequence is by default eps-> pdf, jpg, png, pdf, ...
%    and the JACoW template provides JACpic2v3.eps and JACpic2v3.jpg which
%    might generates errors, therefore PNG and JPG first
%
\makeatletter%
	\ifboolexpr{bool{xetex}}
	 {\renewcommand{\Gin@extensions}{.pdf,%
	                    .png,.jpg,.bmp,.pict,.tif,.psd,.mac,.sga,.tga,.gif,%
	                    .eps,.ps,%
	                    }}{}
\makeatother

% CHECK FOR XeTeX/LuaTeX BEFORE DEFINING AN INPUT ENCODING
% --------------------------------------------------------
%   utf8  is default for XeTeX/LuaTeX
%   utf8  in LaTeX only realises a small portion of codes
%
\ifboolexpr{bool{xetex} or bool{luatex}} % test for XeTeX/LuaTeX
 {}                                      % input encoding is utf8 by default
 {\usepackage[utf8]{inputenc}}           % switch to utf8

\usepackage[USenglish]{babel}

\usepackage[final]{pdfpages}
\usepackage{multirow}
\usepackage{ragged2e}
\usepackage{todonotes}

%
% if BibLaTeX is used
%
\ifboolexpr{bool{jacowbiblatex}}%
 {%
  \addbibresource{jacow-test.bib}
  \addbibresource{biblatex-examples.bib}
 }{}
\listfiles

%
% command for typesetting a \section like word
%
\newcommand\SEC[1]{\textbf{\uppercase{#1}}}

%%
%%   Lengths for the spaces in the title
%%   \setlength\titleblockstartskip{..}  %before title, default 3pt
%%   \setlength\titleblockmiddleskip{..} %between title + author, default 1em
%%   \setlength\titleblockendskip{..}    %afterauthor, default 1em

%\copyrightspace %default 1cm. arbitrary size with e.g. \copyrightspace[2cm]

% testing to fill the copyright space
%\usepackage{eso-pic}
%\AddToShipoutPictureFG*{\AtTextLowerLeft{\textcolor{red}{COPYRIGHTSPACE}}}

\begin{document}

\title{openMMC: An Open-Source Modular Firmware for Board Management}

\author{H. A. Silva\thanks{henrique.silva@lnls.br}, G. B. M. Bruno, LNLS, Campinas, Brazil}

\maketitle

%
\begin{abstract}
  openMMC is an open-source firmware designed for board management in MicroTCA systems.
  It has a modular architecture providing decoupling between application,
  board and microcontroller-specific routines, making it useful as a base for many
  different designs, even the ones using less powerful controllers.
  Despite being developed in a MicroTCA context, the firmware can be easily adapted
  to other hardware platforms and communication protocols.
  The firmware is based on the FreeRTOS operating system, over which each module
  (sensors, LEDs, Payload management, etc) runs its own independent task.
  The OS, despite its reduced footprint, also provides numerous tools for reliable
  communication among the tasks, controlling the board efficiently.
\end{abstract}

\section{Introduction}
LNLS Diagnostic system will use fast digitalizer boards in order to gather the beam position data and calculate the needed orbit corrections. It has been decided that the BPM electronics will adopt MicroTCA.4 standard by PICMG.

AMC Boards, as the application boards are called in the MicroTCA system, must have implemented a Module Management Controller (MMC), which usually is a $\mu$controller responsible for monitoring the board health and acting as a communication channel between the system manager and application using Intel's Intelligent Platform Management Interface (IPMI).

Available open-source MMC implementations are mostly designed for a specific $\mu$controller or board hardware, making it difficult to port the code to other applications. openMMC was created to be a modular and generic firmware, easily portable to other platforms. It runs over FreeRTOS, which gives the developer a wide set of tools to implement complex monitoring functions or advanced hardware control.

Given its modular independent structure, it's possible to use the firmware in applications outside MicroTCA environment with little effort, changing the communication protocol in its lower layers for example.

\section{Motivation}


\section{FreeRTOS}
FreeRTOS is a popular open-source real time operating system for embedded controllers that has already been ported to over than 35 different architectures.

It features a preemptive scheduler that allows the application code to run multiple tasks in parallel with a single core.
The scheduler decides which task will run based on its priority. More important tasks are always executed first, whilst blocking the lower priority ones.
If one or more tasks are on the same priority level, a round-robin time slice method is applied, ensuring that all of them are executed within its time limits.

Most of OS-native tools used in communication between tasks are also implemented on FreeRTOS (e.g. semaphores, software timers, queues).
The developer can also use some unique functions provided, such as direct-to-task notifications, event groups and co-routines.

With the exception of task creation, all of the implemented tools on FreeRTOS can be stripped from its compilation, what (?) drastically reduces the project resource usage,
thus making it possible to use cheaper low-power microcontrollers in the project.

\section{Firmware Structure}
The firmware was structured in order to be easy to upgrade and port to different boards and controllers.
Therefore four different abstraction layers are implemented: \emph{Application}, \emph{Hardware Abstraction}, \emph{Port} and \emph{Driver}, disposed as in Figure ??.

\missingfigure[figwidth=7cm]{openMMC Firmware Structure}

\subsection{Application}
The Application layer holds high-level tasks responsible for deciding which action will be taken based on the information provided by the lower level layers. Most of the functions implemented on this level are state-machine based, regurlarly checking and actuating on the board status.

Each target board has a separated folder in which all its application layer tasks are implemented. It must also list which modules need to be added in the compilation chain.

\subsection{Hardware Abstraction}
In the Hardware Abstraction Layer lies all the functions that interfaces with the board peripheral hardware, implementing the communication protocols when needed. Since any board can benefit from a module in this layer, they're all stored together inside "modules" folder.

A few modules were already implemented and tested with LNLS' AFC board:
\begin{Itemize}

\item IPMI Protocol
\item Watchdog Timer
\item SCANSTA111 JTAG Switch
\item ADN4604 Clock Crossbar Switch
\item DAC AD84XX
\item AT24MAC EEPROM
\item 24xx64 EEPROM
\item Hotswap
\item LM75 Temperature Sensor
\item MAX6642 Temperature Sensor
\item INA220 Voltage and Current Sensor
\item HPM Upgrade
\item PCA9554 I/O Expander
\item Si57X Oscillator
\item UART Debug interface

\end{Itemize}

\subsection{Port}
The Port layer serves as a connection between \emph{Driver} and the upper layers.
Its purpose is to mask all functions provided by the controller drivers so that the upper layers only call generic functions defined in this level, as demonstrated in Figure ??.

\missingfigure[figwidth=7cm]{Port Layer demonstration}

This layer presents a few restrictions regarding its structure. Developers responsible to implement new hardware drivers must follow the port layer functions' signature in order to maintain compatibility throughout the firmware.

Currently, there is only one openMMC maintaned port, which targets LPC17XX chips.
Initial attempts towards a ATxMega128 port have been made by GSI *ref-to-Piotr-repo*, but they have not been certified as compatible with the current openMMC version.

\subsection{Drivers}
The lowest layer is where the controller drivers are implemented, accessing directly the hardware to control the outputs.
Since the firmware upper layers don't call driver functions directly, this can be easily replaced by another implementation when the target controller is changed.

\section{Build Automation}
CMake is one of the most popular Build Automation softwares nowadays and it implements its own scripting language that allows the user to fully customize each build without having to worry about sources dependencies.

Using this tool with openMMC allows developers to fit the firmware to the board hardware just by adding new modules in the compilation list present on each board port folder.
The user is also able to set new compilation flags or link new libraries almost effortless.

\section{Documentation}
The firmware documentation is written using Doxygen style. This way, the documentation format is completely automated and can have many different output formats such as HTML, \LaTeX , XML, RTF, etc.

By hosting the firmware code in GitHub repository, one is able to use a feature called GitHub Pages, in which a small website for generic use can be hosted simply by adding a new branch named \emph{gh-pages} in the main repository.

Using this tool simplifies code documentation update, since it just needs to be regenerated by Doxygen after each change and pulled to its respective GitHub Pages branch.

\section{Continuous Integration}
Every modification in the application layer must be tested across all boards and controllers ports to assert that the firmware remains compatible.

Confirming that the firmware compiles with every variant option becomes very difficult as the ports count increases.
Using a Continuous Integration tool allows the maintainer to run dedicated tests after each commit on the main repository and automatically reports if the lattest change has made any port unusable.
Travis C.I. uses a simple script written in its own language to install the needed toolchains and by tracking the latest commits on the repository branches, it's able to perform build tests on them, ensuring that the latest modification is compatible with all board and controller ports.

Differently from other higher level software projects, embedded software can't be functionally tested with Continuous Integration tools, since they lack the necessary peripheral hardware to perform a quality test.

\section{Code Size}
openMMC has a really small memory footprint, which makes it perfect for embedded environments that traditionally have limited resources.
Table \ref{table:rom_usage} presents openMMC ROM usage after compiling in both minimal and standard configurations for AFC BPM board, using GCC's optimization flag -Os.

\begin{table}[hbt]
  \centering
  \caption{openMMC ROM usage}
  \label{table:rom_usage}
  \begin{tabular}{lcc}
    \toprule
    \textbf{Configuration} & \textbf{Total Size [kB]} \\
    \midrule
    Minimal               & 20.91  \\
    Standard              & 29.69  \\
       \bottomrule
  \end{tabular}
\end{table}

\section{License}
The selected license for this project was GNU General Public License v3.0 or later.

FreeRTOS has a different license based on GPLv3, allowing the user to implement its applcation on top of the OS without having to publish propietary code.

\section{Acknowlegments}
The authors would like to thank P. Miedzik from GSI for the initial discussions in the firmware structure definition and the suggestion on using FreeRTOS as a base for the project.

%\iffalse  % only for "biblatex"
%	\newpage
%	\printbibliography
%
%% "biblatex" is not used, go the "manual" way
%\else
%
%%\begin{thebibliography}{99}   % Use for  10-99  references
%\begin{thebibliography}{9} % Use for 1-9 references
%
%\bibitem{jacow-help}
%	JACoW,
%	\url{http://www.jacow.org}
%
%\bibitem{IEEE}
%	\textit{IEEE Editorial Style Manual},
%	IEEE Periodicals, Piscataway,
%	NJ, USA, Oct. 2014, pp. 34--52.
%\end{thebibliography}
%
%\fi

%\definecolor{jgreen}{cmyk}{0.81, 0.00, 0.97, 0.00}
%\definecolor{jred}{cmyk}  {0.00, 0.99, 1.00, 0.00}
%\definecolor{jgrepc}{cmyk}{0.74, 0.05, 1.00, 0.00}
%\definecolor{jblue}{cmyk} {0.87, 0.54, 0.00, 0.00}
%\definecolor{jvio}{cmyk}  {0.41, 0.82, 0.00, 0.00}
%\definecolor{jbook}{cmyk} {0.28, 0.88, 0.79, 0.25}
%\definecolor{jrept}{cmyk} {0.07, 0.70, 1.00, 0.00}
%\definecolor{jmanu}{cmyk} {0.28, 0.77, 1.00, 0.23}
%\definecolor{junpu}{cmyk} {0.00, 0.83, 0.65, 0.00}

\end{document}
